\section{Формулировка индивидуального задания}

Вариант №25. Необходимо спроектировать и реализовать на языке C две программы, позволяющие вычислять
значения некоторой заданной функции. Первая программа должна принимать на вход значение аргумента \texttt{X} и количество членов ряда \texttt{N}, необходимое для проведения вычислений, вычислять значение функции с помощью суммы \texttt{N} членов ряда и выводить вычисленные значения.
Вторая программа должна принимать на вход значение аргумента \texttt{X} и точность \texttt{EPS}, необходимую для проведения вычислений, вычислять значение суммы ряда с заданной точностью \texttt{EPS} и возвращать вычисленное значение, а также количество членов ряда, потребовавшееся для обеспечения заданной точности.
При этом, в обеих программах должно осуществляться вычисление значения функции не только
при помощи разложения в ряд, но и с использованием функций стандартной библиотеки.

\[ \LARGE e^x\,\cos(x) = \sum_{n=0}^\infty \frac{2^{\frac{n}{2}}\,\cos(\frac{\pi n}{4})}{n!} x^n\]
